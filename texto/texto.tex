\documentclass[10pt,twocolumn,letterpaper]{article}


\usepackage{times}
\usepackage{epsfig}
\usepackage{graphicx}
\usepackage{amsmath}
\usepackage{amssymb}

% USER PACKAGES
%\usepackage[spanish]{babel}
\usepackage[utf8]{inputenc}

% Include other packages here, before hyperref.
\usepackage[breaklinks=true,bookmarks=false]{hyperref}



\begin{document}
%\title{}

\begin{abstract}
Se construyo dos redes neuronales convolucionales binarias, una para caras de personas y otra para sombreros, con el fin de localizar caras y sombreros de forma individual en una imagen en particular. Se proceso los resultados de las dos redes neuronales y con cada respuesta se construyo un mapa lógico en el cual se puede localizar la zona donde esta presente el objeto de interés, caras o sombreros. Posteriormente se procesan los dos mapas y del procesamiento se puede etiquetar la imagen indicando si las caras presentes en la imagen están usando o no sombreros.
\end{abstract}

\section{Introducción}
Este documento se organizo de la siguiente forma: En la sección 2 se analizan trabajos previos de reconocimiento del contenido de una imagen. En la sección 3 se describen los materiales y metodologías usados para abordar el proyecto. En la sección 4 se profundiza en la metodología usada y en la sección 5 se presentan los resultados obtenidos de esta metodología. En la sección 6 se presentan las conclusiones de los resultados obtenidos.


\section{Trabajo Previo}

 
Google Cloud Vision API 
Image Recognition Tensorflow
Integrating Humans and Computers for Image and Video Understanding 
How Will Google "Read" \& Rank Your Images in the Near Future

\begin{itemize}
\item Google Cloud Vision API 
\item Image Recognition Tensorflow
\item Integrating Humans and Computers for Image and Video Understanding 
\item How Will Google "Read" \& Rank Your Images in the Near Future
\end{itemize}









%https://cloud.google.com/vision/
%https://www.tensorflow.org/tutorials/image_recognition/
%http://www3.cs.stonybrook.edu/~cvl/gaze.html
%https://cognitiveseo.com/blog/6511/will-google-read-rank-images-near-future/


%https://www.ibm.com/watson/developercloud/visual-recognition.html
%https://www.technologyreview.com/s/600889/google-unveils-neural-network-with-superhuman-ability-to-determine-the-location-of-almost/
%https://medium.com/@ageitgey/machine-learning-is-fun-part-3-deep-learning-and-convolutional-neural-networks-f40359318721#.8iolcksq0
%http://www.cse.buffalo.edu/~jcorso/r/snippets.semantic_segmentation.html

\section{Materiales y Métodos}
\subsection{Materiales}
Las imágenes de las caras y otras imágenes diferentes a caras fueron obtenidas de "Labeled Faces in the Wild Home y Caltech 101".
Las imágenes de los sombreros y otras imágenes diferentes a sombreros fueron obtenidas de "ImageNet"
Se uso la biblioteca TensorFlow para el lenguaje de programación Python 3.5

\subsection{Métodos}
Se uso una metodología basada en redes neuronales convolucionales tomando como base la implementación de Cifar-10, el cual se encarga de clasificar y etiquetar imágenes RGB de 32x32 de 10 clases diferentes. 
Debido a que no es viable entrenar una red neuronal con imágenes sin ningún tipo de procesamiento por la gran cantidad de píxeles que implica una imagen, es necesario reducir esta cantidad sin perder, en gran medida, la información importante que brindan todo el conjunto de datos, por lo tanto aplicamos la convolución para resaltar y extraer la información importante de la imagen y eliminar píxeles que no brinden información de interesante, sin embargo la matriz resultante aun es grande para la red neuronal. Para reducir las dimensiones de esta matriz se recurre a la operación no lineal max-pooling que consiste en tomar los valores mas significativos en múltiples ventanas de determinada dimensión de la matriz procesada y como resultado obtenemos una matriz mucho mas reducida y adecuada para el procesamiento en la red neuronal. Es decir, una red neuronal convolucional consiste en dos procesos, el primero se encarga de reducir las dimensiones de los datos a partir de convoluciones y otras operaciones no lineales y el segundo consiste en la red neuronal alimentada con los datos del primer proceso.

La arquitectura de Cifar-10 consiste en una capa de convolución seguida de una operación max-pooling y una normalización, luego otra capa de convolución, una normalización y un max-pooling y finalmente el procesamiento de la red neuronal.

%https://www.tensorflow.org/tutorials/deep_cnn/

\section{Experimentos}

\section{Resultados}

\section{Conclusiones}


\end{document}
