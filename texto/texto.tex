\documentclass[10pt,twocolumn,letterpaper]{article}


\usepackage{times}
\usepackage{epsfig}
\usepackage{graphicx}
\usepackage{amsmath}
\usepackage{amssymb}

% USER PACKAGES
%\usepackage[spanish]{babel}
\usepackage[utf8]{inputenc}

% Include other packages here, before hyperref.
\usepackage[breaklinks=true,bookmarks=false]{hyperref}



\begin{document}
%\title{}

\begin{abstract}
Se construyo dos redes neuronales convolucionales binarias, una para caras de personas y otra para sombreros, con el fin de localizar caras y sombreros de forma individual en una imagen en particular. Se proceso los resultados de las dos redes neuronales y con cada respuesta se construyo un mapa lógico en el cual se puede localizar la zona donde esta presente el objeto de interés, caras o sombreros. Posteriormente se procesan los dos mapas y del procesamiento se puede etiquetar la imagen indicando si las caras presentes en la imagen están usando o no sombreros.
\end{abstract}

\section{Introducción}
Este documento se organizo de la siguiente forma: En la sección 2 se analizan trabajos previos de reconocimiento del contenido de una imagen. En la sección 3 se describen los materiales y metodologías usados para abordar el proyecto. En la sección 4 se profundiza en la metodología usada y en la sección 5 se presentan los resultados obtenidos de esta metodología. En la sección 6 se presentan las conclusiones de los resultados obtenidos.


\section{Trabajo Previo}

 
Google Cloud Vision API 
Image Recognition Tensorflow
Integrating Humans and Computers for Image and Video Understanding 
How Will Google "Read" \& Rank Your Images in the Near Future

\begin{itemize}
\item Google Cloud Vision API 
\item Image Recognition Tensorflow
\item Integrating Humans and Computers for Image and Video Understanding 
\item How Will Google "Read" \& Rank Your Images in the Near Future
\end{itemize}









%https://cloud.google.com/vision/
%https://www.tensorflow.org/tutorials/image_recognition/
%http://www3.cs.stonybrook.edu/~cvl/gaze.html
%https://cognitiveseo.com/blog/6511/will-google-read-rank-images-near-future/


%https://www.ibm.com/watson/developercloud/visual-recognition.html
%https://www.technologyreview.com/s/600889/google-unveils-neural-network-with-superhuman-ability-to-determine-the-location-of-almost/
%https://medium.com/@ageitgey/machine-learning-is-fun-part-3-deep-learning-and-convolutional-neural-networks-f40359318721#.8iolcksq0
%http://www.cse.buffalo.edu/~jcorso/r/snippets.semantic_segmentation.html

\section{Materiales y Métodos}
\subsection{Materiales}
Las imágenes de las caras y otras imágenes diferentes a caras fueron obtenidas de "Labeled Faces in the Wild Home y Caltech 101".
Las imágenes de los sombreros y otras imágenes diferentes a sombreros fueron obtenidas de "ImageNet"
Se uso la biblioteca TensorFlow para el lenguaje de programación Python 3.5

\subsection{Métodos}
Se uso una metodología basada en redes neuronales convolucionales tomando como base la implementación de Cifar-10, el cual se encarga de clasificar y etiquetar imágenes RGB de 32x32 de 10 clases diferentes. La arquitectura de Cifar-10 consiste en capas de convolución y operaciones no lineales cuyo resultado es procesado por una red neuronal, el procedimiento se describe en: %https://www.tensorflow.org/tutorials/deep_cnn/

\section{Experimentos}

\section{Resultados}

\section{Conclusiones}


\end{document}
